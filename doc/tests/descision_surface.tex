\def\setmeshr#1{
  \def\meshr{#1}
}
\def\meshr{1pt}

\pgfooclass{meshgrid visualizer}{
  % Stores the name of the visualizer. This is needed for
  % filtering and configuration
  \attribute name;

  % The constructor. Just setup the attribute.
  \method meshgrid visualizer(#1) { \pgfooset{name}{#1} }

  % Connect to visualize signal.
  \method default connects() {
    \pgfoothis.get handle(\me)
    \pgfkeysvalueof{/pgf/data visualization/obj}.connect(
      \me,visualize,visualize datapoint signal)
  }

  % This method is invoked for each data point. It checks
  % whether the data point belongs to the correct
  % visualizer and, if so, calls the macro \dovisualization
  % to do the actual visualization.
  \method visualize() {
    \pgfdvfilterpassedtrue
    \pgfdvnamedvisualizerfilter
    \ifpgfdvfilterpassed
      \dovisualization
    \fi
  }
}

\def\dovisualization{
  \pgfkeysvalueof{%
    /data point/\pgfoovalueof{name}/execute at begin%
  }

  \pgfpointdvdatapoint
  \pgfgetlastxy{\macrox}{\macroy}

  \pgfmathsetmacro\xlow {\macrox - \meshr}
  \pgfmathsetmacro\ylow {\macroy - \meshr}
  \pgfmathsetmacro\xhigh{\macrox + \meshr}
  \pgfmathsetmacro\yhigh{\macroy + \meshr}

  \pgfpathrectanglecorners{\pgfpoint{\xlow}{\ylow}}
                          {\pgfpoint{\xhigh}{\yhigh}}

  \pgfkeysvalueof{%
    /data point/\pgfoovalueof{name}/execute at end%
  }
}
\tikzdatavisualizationset{
  visualize as meshgrid/.style={
    new object={
      when=after survey,
      store=/tikz/data visualization/visualizers/#1,
      class=meshgrid visualizer,
      arg1=#1
    },
    new visualizer={#1}{%
      color=visualizer color,
      every path/.style={fill,draw,opacity=0.5},
    }{},
    /data point/set=#1
  },
  visualize as meshgrid/.default=meshgrid
}

\setmeshr{0.5pt}

\def\visualizeds#1{
  \begin{tikzpicture}
    \datavisualization [
      scientific axes=clean,
      visualize as meshgrid/.list={mesh0, mesh1},
      visualize as scatter/.list={data0, data1},
      x axis={label={$x_0$}},
      y axis={label={$x_1$}},
      mesh0={style={color=red!50}},
      mesh1={style={color=blue!50}},
      data0={style={mark=o, mark size=0.1pt,
        visualizer color=red}},
      data1={style={mark=o, mark size=0.1pt,
        visualizer color=blue}},
    ]
    data[headline={x, y}, read from file=#1.mesh_0.csv,
      set=mesh0]
    data[headline={x, y}, read from file=#1.mesh_1.csv,
      set=mesh1]
    data[headline={x, y}, read from file=#1.data_0.csv,
      set=data0]
    data[headline={x, y}, read from file=#1.data_1.csv,
      set=data1]
    ;
  \end{tikzpicture}
}

% descision surface figure {{{
\begin{figure*}
  \begin{comment}
  \begin{subfigure}[b]{0.3\textwidth}
    \centering
    \scalebox{0.75}{\visualizeds{tests/data/e2_ls2}}
    \caption{$\tau_{|X|} = 2$, $N = 2$}
    \label{fig:x2_n2}
  \end{subfigure}
  \begin{subfigure}[b]{0.3\textwidth}
    \centering
    \scalebox{0.75}{\visualizeds{tests/data/e5_ls2}}
    \caption{$\tau_{|X|} = 2$, $N = 5$}
  \end{subfigure}
  \begin{subfigure}[b]{0.3\textwidth}
    \centering
    \scalebox{0.75}{\visualizeds{tests/data/e10_ls2}}
    \caption{$\tau_{|X|} = 2$, $N = 10$}
    \label{fig:x2_n10}
  \end{subfigure}

  \vspace{1cm}

  \begin{subfigure}[b]{0.3\textwidth}
    \centering
    \scalebox{0.75}{\visualizeds{tests/data/e2_ls5}}
    \caption{$\tau_{|X|} = 5$, $N = 2$}
  \end{subfigure}
  \begin{subfigure}[b]{0.3\textwidth}
    \centering
    \scalebox{0.75}{\visualizeds{tests/data/e5_ls5}}
    \caption{$\tau_{|X|} = 5$, $N = 5$}
  \end{subfigure}
  \begin{subfigure}[b]{0.3\textwidth}
    \centering
    \scalebox{0.75}{\visualizeds{tests/data/e10_ls5}}
    \caption{$\tau_{|X|} = 5$, $N = 10$}
  \end{subfigure}

  \vspace{1cm}

  \begin{subfigure}[b]{0.3\textwidth}
    \centering
    \scalebox{0.75}{\visualizeds{tests/data/e2_ls10}}
    \caption{$\tau_{|X|} = 10$, $N = 2$}
    \label{fig:x10_n2}
  \end{subfigure}
  \begin{subfigure}[b]{0.3\textwidth}
    \centering
    \scalebox{0.75}{\visualizeds{tests/data/e5_ls10}}
    \caption{$\tau_{|X|} = 10$, $N = 5$}
  \end{subfigure}
  \begin{subfigure}[b]{0.3\textwidth}
    \centering
    \scalebox{0.75}{\visualizeds{tests/data/e10_ls10}}
    \caption{$\tau_{|X|} = 10$, $N = 10$}
    \label{fig:x10_n10}
  \end{subfigure}
  \end{comment}

  \begin{flushright}
    \begin{tikzpicture}
      \begin{scope}[label distance=2pt]
        \node[color=red, circle, draw,fill,
          label=right:label 0, inner sep=1.5pt] at (0,0)
            (l_zero) {};
        \node[color=blue, circle,draw,fill,
          label=right:label 1, below=0.2 of l_zero,
          inner sep=1.5pt] (l_one) {};
      \end{scope}

      \node[color=red!50,opacity=0.5,
        label=right:predicted 0, below=0.2 of l_one, fill,
        draw] (m_zero) {};
      \node[color=blue!50,opacity=0.5,below=0.2 of m_zero,
        label=right:predicted 1, fill, draw] (m_one) {};
    \end{tikzpicture}
  \end{flushright}
  \caption{The descision surfaces of differently configured
    PCFs. The threshold $\tau_{|X|}$ and the amount of
    Trees $N$ of the PCF are different for each Figure,
    \ref{fig:x2_n2} - \ref{fig:x10_n10}.}
  \label{fig:descision_surface}
\end{figure*}
% }}}
