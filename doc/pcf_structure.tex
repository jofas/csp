\section{Partial Classification Forest}
\label{sec:pcf}

The PCF has two parameters, (\romannumeral 1) $N$ and
(\romannumeral 2) an array with $N$ pointers.
$N$ is the amount of Trees the PCF maintains.
Initially the pointers inside the array are references to
Nil nodes.

The PCF offers the same two operations a Tree has, FIT
(Algorithm~\ref{alg:pcf_fit}) and PREDICT (Algorithm~%
\ref{alg:pcf_pred}), both abstractions to the equivalent
Tree operations.

Once FIT is executed, the pointers are references to the
roots of fitted Tree instances.

Both FIT and PREDICT can be implemented as multithreaded
operations as long as $\gamma$ is threadsafe, since the
Tree instances are independent of each other and the shared
parameters $X$, $y$ (FIT) and $x$ (PREDICT) are read only,
making synchronization unnecessary.

FIT first computes $\beta_X$ which has a time complexity of
$\mathcal{O}(|X[0]| * |X|)$, $|X[0]|$ denoting the amount
of dimensions the feature space has.

After that the Tree's FIT operation is called $N$ times,
which means the PCF's FIT operation has a worst case time
complexity of $\mathcal{O} (N * (2^{\tau_h + 1} - 1) *
(|X| + \mathcal{O}(\gamma)) + |X[0]| * |X|)$
(cmp.~\ref{subsec:fit}).

% {{{
\begin{algorithm}
  \caption{: FIT($\Pi, X, y, \gamma, \tau_{l},
    \tau_{|X|}, \tau_{h}$)}%
  \label{alg:pcf_fit}
  The PCF's FIT operation.

  Inputs:

    \begin{tabu}{llX}
    $\Pi$ &$-$ &a PCF instance,\\
    $X$ &$-$ &input data,\\
    $y$ &$-$ &labels of X,\\
    $\gamma$ &$-$ &function returning a predictor and its
      loss,\\
    $\tau_{l}$ &$-$ &loss threshold,\\
    $\tau_{|X|}$ &$-$ &threshold for the size of X,\\
    $\tau_{h}$ &$-$ &height limit of the Tree
    \end{tabu}

  Output: void

  \noindent\rule{\linewidth}{0.4pt}

  \begin{algorithmic}[1]
    \STATE compute $\beta_X$
    \FORALL{$\Theta \in \Pi$.trees}
      \STATE{FIT($\Theta$, $X$, $y$, 0, $\beta_X$, \dots)}
    \ENDFOR
  \end{algorithmic}
\end{algorithm}
% }}}

The PCF's PREDICT operation first initializes an array with
$N$ elements (Algorithm~\ref{alg:pcf_pred}, line 1). Each
Tree instance fills one element of the array with its
prediction. After that the PCF's PREDICT operation takes
the label predicted most and returns it as its prediction
for the observation $x$ (Algorithm~\ref{alg:pcf_pred},
lines 5, 6).

The worst case time complexity of the PCF's PREDICT
operation is $\mathcal{O}(N * (\tau_h + \mathcal{O}
(\text{predictor})) + N)$, since a Tree's PREDICT operation
is executed $N$ times, plus the most predicted label
must be determined, which is $\mathcal{O}(N)$.

% {{{
\begin{algorithm}
  \caption{: PREDICT($\Pi, x$)}
  \label{alg:pcf_pred}
  The PCF's PREDICT operation.

  Inputs:

    \begin{tabu}{llX}
    $\Pi$ &$-$ &a PCF instance,\\
    $x$ &$-$ &an observation, \\
    \end{tabu}

  Output: the predicted label or $\Lambda$

  \noindent\rule{\linewidth}{0.4pt}

  \begin{algorithmic}[1]
    \STATE predictions $\leftarrow [\Lambda; N]$
    \FOR{$i = 1$ \TO $N$}
      \STATE{predictions$[i]$ = PREDICT($\Pi$.trees$[i]$,
        $x$, $0$)}
    \ENDFOR
    \STATE determine $l_{max}$, the label predicted most
    \RETURN $l_{max}$
  \end{algorithmic}
\end{algorithm}
% }}}
