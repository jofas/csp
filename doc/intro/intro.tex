\documentclass[12pt]{article}
\usepackage{amsmath}
\usepackage{amssymb}
\usepackage[colorlinks=true, linkcolor=black,citecolor=black]{hyperref} % Links
\usepackage{makeidx} % Indexierung
\usepackage{siunitx}
\usepackage[ngerman]{babel} % deutsche Sonderzeichen
\usepackage[utf8]{inputenc}
\usepackage{geometry} % Dokumentendesign wie Seiten- oder Zeilenabstand bestimmen
\usepackage[toc,page]{appendix}

% Graphiken
\usepackage{tikz}
\usepackage{pgfplots}
\usepackage{pgfcore}
\usepackage{pgfopts}
\usepackage{pgfornament}
\usepackage{pgf}
\usepackage{ifthen}
\usepackage{booktabs}

% Tabellen
\usepackage{tabu}
\usepackage{longtable}
\usepackage{colortbl} % Tabellen faerben
\usepackage{multirow}
\usepackage{diagbox} % Tabellenzelle diagonal splitten

\usepackage{xcolor} % Farben
\usepackage[framemethod=tikz]{mdframed} % Hintergrunderstellung
\usepackage{enumitem} % Enumerate mit Buchstaben nummerierbar machen
\usepackage{pdfpages}
\usepackage{listings} % Source-Code darstellen
\usepackage{eurosym} % Eurosymbol
\usepackage[square,numbers]{natbib}
\usepackage{here} % figure an richtiger Stelle positionieren
\usepackage{verbatim} % Blockkommentare mit \begin{comment}...\end{comment}
\usepackage{ulem} % \sout{} (durchgestrichener Text)

% BibLaTex
\bibliographystyle{alpha}

% Aendern des Anhangnamens (Seite und Inhaltsverzeichnis)
\renewcommand\appendixtocname{Anhang}
\renewcommand\appendixpagename{Anhang}

% mdframed Style
\mdfdefinestyle{codebox}{
	linewidth=2.5pt,
	linecolor=codebordercolor,
	backgroundcolor=codecolor,
	shadow=true,
	shadowcolor=black!40!white,
	fontcolor=black,
	everyline=true,
}

% Seitenabstaende
\geometry{left=15mm,right=15mm,top=10mm,bottom=20mm}

% TikZ Bibliotheken
\usetikzlibrary{
    arrows,
    arrows.meta,
    decorations,
    backgrounds,
    positioning,
    fit,
    petri,
    shadows,
    datavisualization.formats.functions,
    calc,
    shapes,
    shapes.multipart
}

\pgfplotsset{width=7cm,compat=1.15}

\definecolor{codecolor}{HTML}{EEEEEE}
\definecolor{codebordercolor}{HTML}{CCCCCC}

% Standardeinstellungen fuer Source-Code
\lstset{
    language=C,
    breaklines=true,
    keepspaces=true,
    keywordstyle=\bfseries\color{green!70!black},
    basicstyle=\ttfamily\color{black},
    commentstyle=\itshape\color{purple},
    identifierstyle=\color{blue},
    stringstyle=\color{orange},
    showstringspaces=false,
    rulecolor=\color{black},
    tabsize=2,
    escapeinside={\%*}{*\%},
}

%\input{libuml}
%\input{liberm}

\title{Partial classification of a binary labeled dataset
  (working title)}

\author{Jonas Fa{\ss}bender}
\date{}

\begin{document}

\maketitle

\section{Problem}

Supervised learning is the task of approximating the
unknown function $y = f(x)$ where $x$ generates
$y$. Both $x$ and $y$ can be any object. If $y$ is from a
finite set this learning problem is called classification.

A classifier learns $h(x)$ based on a set $T$ containing
$n$ example tuples:
\begin{align*}
T := \{(x_1, y_1), (x_2, y_2), \dots ,(x_n,y_n)\}
\end{align*}
referred to as the training set.\cite{ki}

A classifier is trained to approximate $f(x)$ with $h(x)$
based on $T$, which makes the accuracy $a$,
$0 \leq a \leq 1$ of $h(x)$ (how well $h(x)$ generalizes
and predicts the corresponding $y$ to unseen $x$ from a
test set $T'$ similar to $T$) dependent on the quality of
$T$.\cite{tc_data}

For some datasets the data quality is not high
enough to be able to find $h(x)$ with a sufficient
accuracy $a(h(x))$. In this case sufficient means that
$a(h(x))$ is greater than or equal to a context dependent
value $s$ representing a threshold $s \leq a(h(x))$.

I am trying to find a method that first clusters a dataset
with which I can only train classifiers that build
insufficient $h(x)$ and afterwards finds a subset
$P_s \subseteq P$ of these clusters or partitions $P$ of
the dataset so that for each partition in $P_s$ exists a
classifier with a sufficient $h(x)$. My goal is to maximize
the space $P_s$ covers.

\section{Next steps}

To start this project I want to begin with a literature
research focusing on the following topics:

\begin{itemize}

  \item Common clustering methods, especially the k-means
        algorithm

  \item Nearest Neighbor algorithms, Voronoi Cells, LSH
        (Local Sensitive Hashing)

  \item Isolation Forest and ensemble-based classification
        methods

  \item Randomized methods (Monte Carlo methods)

\end{itemize}

\bibliography{intro}

\end{document}
