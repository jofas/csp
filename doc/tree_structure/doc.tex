\section{The Tree structure}
\label{sec:tree}

A Tree generated by the PCF is a binary search tree
structure similar to k-d trees. Its purpose is to randomly
generate disjoint partitions of a feature space.

A Tree has two types of nodes, non-leaf nodes, here denoted
as Nodes and leaf nodes denoted as Leafs. It provides two
operations: (\romannumeral 1) FIT, initializing the Tree
and (\romannumeral 2) PREDICT, returning a label for an
oberservation.

The Node structure contains three properties:
(\romannumeral 1) a split value; (\romannumeral 2) a left
and (\romannumeral 3) a right successor, both references to
either another Node or a Leaf.

A Leaf on the other hand, is the structure
representing a partition of the feature space, having the
following properties: (\romannumeral 1) active: a boolean
value deciding whether the partition's quality, determined
during the FIT operation, is equal or better than the
defined threshold or not; (\romannumeral 2) optionally a
predictor which is used to classify oberservations during
the PREDICT operation. Only if a Leaf's active property is
true, a predictor must be provided. A Leaf also has two
vectors with arbitrary length as properties:
(\romannumeral 3) a vector containing the observations of
the dataset used in FIT, which are laying inside the
partition and (\romannumeral 4) their inherent labels.

During the FIT operation a Tree contains a third type of
node, Nil. Nil is used to initialize Trees
and the left and right successor of a Node. These nodes are
transformed during FIT to either a Node or a Leaf, so after
the FIT operation a Tree does not contain Nil nodes
anymore. A Nil node does not have any properties.

\subsection{The FIT operation}

The FIT operation constructs a Tree, based on a dataset
split in observations ($X$) and their labels ($y$).
Algorithm~\ref{alg:fit} shows how FIT recursively builds a
Tree, which is at the beginning a pointer to a Nil node.

The most important parameter passed to FIT is $\gamma$.
$\gamma$ is a function returning (\romannumeral 1) a
predictor and (\romannumeral 2) the loss of it. Otherwise
$\gamma$ is treated as a black box by the PCF, so what the
predictor is and how its loss is calculated are not
relevant to the PCF, as long as the predictor is callable
and returns an element from the label set when called
(Algorithm~\ref{alg:pred}, line 9). The loss returned
by $\gamma$ gets compared to the quality threshold
$\tau_l$. Is the loss $\leq \tau_l$ the predictor is good
enough and $\Theta$ is transformed to an active Leaf
(Algorithm~\ref{alg:fit}, lines 2, 3).

There are two other thresholds besides $\tau_l$,
$\tau_{|X|}$, $\tau_h$. Both regulate the behaviour of a
Tree's growth. $\tau_{|X|}$ defines a minimum amount of
observations a Leaf must contain. One can easily imagine,
without $\tau_{|X|}$ or $\tau_{|X|} = 0$ a Tree would
never stop growing, since FIT would continue to split empty
partitions, trying to find a smaller partition which would
be predictable, even though no predictor could be
generated without observations to train it on.

$\tau_h$ further regulates the maximum path length of a
Tree. It is necessary besides $\tau_{|X|}$, because of the
following scenario: be $\tau_{|X|} = 2$ and there are two
equal observations in the dataset, but both having a
different label than the other one. Now $\gamma$, passed
$X$ containing only those two identical observations,
returns a predictor with a loss $> \tau_l$. Since $|X|$ is
still not smaller than $\tau_{|X|}$ FIT would continue
trying to separate the two inseperable observations. To
prevent such a szenario $\tau_h$ tells FIT to stop before
the Tree's height, the amount of edges of the longest path,
would exceed $\tau_h$. The path length of the Tree's root
to $\Theta$ is passed as a parameter $h$ to FIT.

Now, if neither $\tau_l$ is exceeded nor $\tau_{|X|}$ or
$\tau_h$ is violated, FIT performs a split and transforms
$\Theta$ to a Node (Algorithm~\ref{alg:fit}, lines 7ff).
The dimension the split is performed on is chosen in a
cyclic manner, a practise also applied to k-d trees.%
~\cite{Brown2015kdtree}
But rather than chosing the splitting value at the median
of the observations in the dimension, which is done in
order to construct balanced k-d trees, the splitting value
is random.\cite{Brown2015kdtree}

Afterwards FIT is recursively applied to the two new
partitions (Algorithm~\ref{alg:fit}, lines 11, 12).

\begin{algorithm}
  \caption{: FIT($\Theta, X, y, h, \beta_X, \gamma,
    \tau_{l}, \tau_{|X|}, \tau_{h}$)}%
  \label{alg:fit}
  Inputs:

    \begin{tabu}{llX}
    $\Theta$ &$-$ &a pointer to a Nil node; initially
      pointing to the root node of an empty Tree,\\
    $X$ &$-$ &input data,\\
    $y$ &$-$ &labels of X,\\
    $h$ &$-$ &height of the Tree; initially $h = 0$,\\
    $\beta_X$ &$-$ &lower and upper boundries of every
      dimension of X,\\
    $\gamma$ &$-$ &function returning a predictor an its
      quality,\\
    $\tau_{l}$ &$-$ &quality threshold,\\
    $\tau_{|X|}$ &$-$ &threshold for the size of X,\\
    $\tau_{h}$ &$-$ &height limit of the Tree
    \end{tabu}

  Output: void

  \noindent\rule{\linewidth}{0.4pt}

  \begin{algorithmic}[1]
    \STATE predictor, loss $\leftarrow \gamma(X, y)$
    \IF{loss $\leq \tau_{l}$}
      \STATE $\Theta \leftarrow$ LEAF(\TRUE, predictor,
         $X$, $y$)
    \ELSIF{$h > \tau_{h}$ \OR $|X| < \tau_{|X|}$ \OR
        loss $> \tau_{l}$}
      \STATE $\Theta \leftarrow$ LEAF(\FALSE, predictor,
        $X$, $y$)
    \ELSE
      \STATE dimension $\leftarrow h$ mod $|X[0]|$
      \STATE split $\leftarrow$ RANDOM($\beta_X[$dimension
        $]$)
      \STATE $\Theta \leftarrow$ NODE(split, NIL, NIL)
      \STATE split $X$, $y$ and $\beta_X$ into
        $X'$, $X''$, $y'$, $y''$, $\beta_X'$, $\beta_X''$
      \STATE FIT($\Theta$.left, $X'$, $y'$, $h + 1$,
        $\beta_X'$, \dots)
      \STATE FIT($\Theta$.right, $X''$, $y''$, $h + 1$,
        $\beta_X''$, \dots)
    \ENDIF
  \end{algorithmic}
\end{algorithm}


\subsection{The PREDICT operation}

The PREDICT operation traverses a Tree until it encounters
a Leaf. If the Leaf is active a label to a provided
observation $x$ is returned by the predictor
property of the Leaf, otherwise $\Lambda$ is returned
(Algorithm~\ref{alg:pred}).

$\Lambda$ must not be an element of the label set.

\begin{algorithm}
  \caption{: PREDICT($\Theta, x, h$)}%
  \label{alg:pred}
  Inputs:

    \begin{tabu}{llX}
    $\Theta$ &$-$ &a Tree node; initially pointing to the
      root of the Tree,\\
    $x$ &$-$ &an observation,\\
    $h$ &$-$ &height of the Tree; initially $h = 0$
    \end{tabu}

  Output: the predicted label or $\Lambda$

  \noindent\rule{\linewidth}{0.4pt}

  \begin{algorithmic}[1]
    \IF{TYPE($\Theta$) is Node}
      \STATE dimension $\leftarrow h$ mod $|x|$
      \IF{$x[dimension] \leq \Theta$.split}
        \STATE PREDICT($\Theta$.left, $x$, $h + 1$)
      \ELSE
        \STATE PREDICT($\Theta$.right, $x$, $h + 1$)
      \ENDIF
    \ELSIF{$\Theta$.active}
      \RETURN $\Theta$.predictor($x$)
    \ELSE
      \RETURN $\Lambda$
    \ENDIF
  \end{algorithmic}
\end{algorithm}

