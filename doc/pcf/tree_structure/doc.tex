\section{The Tree structure}

A Tree generated by the PCF is a binary search tree
structure similar to k-d trees. Its purpose is to randomly
generate disjoint partitions of a feature space.

A Tree has two types of nodes, non-leaf nodes, here denoted
as Nodes and leaf nodes denoted as Leafs. It provides two
operations: (\romannumeral 1) FIT, initializing the Tree
and (\romannumeral 2) PREDICT, returning a label for an
oberservation.

The Node structure contains three properties:
(\romannumeral 1) a split value; (\romannumeral 2) a left
and (\romannumeral 3) a right successor, both references to
either another Node or a Leaf.

A Leaf on the other hand, is the structure
representing a partition of the feature space, having the
following properties: (\romannumeral 1) active, a boolean
value deciding whether the partition's quality, determined
during the FIT operation, is equal or better than the
defined threshold or not; (\romannumeral 2) optionally a
predictor which is used to classify oberservations during
the PREDICT operation. Only if a Leaf's active property is
true, a predictor must be provided. A Leaf also has two
vectors with arbitrary length as properties:
(\romannumeral 3) a vector containing the observations of
the dataset used in FIT, which are laying inside the
partition and (\romannumeral 4) their inherent labels.

During the FIT operation a Tree contains a third type of
node, Nil. Nil is used to initialize Trees
and the left and right successor of a Node. These nodes are
transformed during FIT to either a Node or a Leaf, so after
the FIT operation a Tree does not contain Nil nodes
anymore. A Nil node does not have any properties.

\subsection{The FIT operation}
\label{subsec:fit}

The FIT operation constructs a Tree based on a dataset
split into observations ($X$) and their labels ($y$).
Algorithm~\ref{alg:fit} shows how FIT recursively builds a
Tree from a pointer to a Nil node.

The most important parameter passed to FIT is $\gamma$.
$\gamma$ is a function returning (\romannumeral 1) a
classifier and (\romannumeral 2) the loss of it. Otherwise
$\gamma$ is treated as a black box by the PCF, so what the
classifier is and how its loss is calculated are not
relevant to the PCF, as long as the classifier is callable%
\footnote{There could be another interface for the
  classifier, for example a predict method similar to the
  scikit-learn library.\cite{sklearn_api}}
and returns an element from the label set when being
called (Algorithm~\ref{alg:pred}, line 9). The loss
returned by $\gamma$ gets compared to the quality threshold
$\tau_l$. Is the loss $\leq \tau_l$ the classifier returned
by $\gamma$ is considered good enough and $\Theta$ is
transformed into an active Leaf (Algorithm~\ref{alg:fit},
lines 2, 3).

There are two other thresholds besides $\tau_l$:
$\tau_{|X|}$ and $\tau_h$. Both regulate the behavior of a
Tree's growth. $\tau_{|X|}$ defines a minimum amount of
observations a Leaf must contain. Without $\tau_{|X|}$ or
$\tau_{|X|} = 0$ a Tree would never stop growing, since FIT
would continue to split empty partitions trying to find a
smaller partition which could be predictable, even though
no classifier can be generated without observations to
train it on.

$\tau_h$ further regulates the maximum path length of a
Tree. It is necessary besides $\tau_{|X|}$: be
$\tau_{|X|} = 2$ and there are two equal observations in
the dataset, both having a different label than the other
one. $\gamma$---passed $X$ containing only those two
identical observations---returns a classifier with a
loss $> \tau_l$. Since $|X|$ is still not smaller than
$\tau_{|X|}$ FIT would continue trying to separate the two
inseparable observations. To prevent such a scenario
$\tau_h$ regulates FIT to stop before the Tree's height%
---the amount of edges of the longest path---would
exceed $\tau_h$. The path length of the Tree's root to
$\Theta$ is passed as a parameter $h$ to FIT. $\tau_h$ also
provides a way to further regulate the time complexity of
FIT and PREDICT.

FIT performs a split and transforms $\Theta$ to a Node
(Algorithm~\ref{alg:fit}, lines 7ff), if neither the
classifier's loss matches $\tau_l$ nor $\tau{|X|}$ or
$\tau_h$ is violated. The dimension the split is performed
on is chosen in a cyclic manner, a practice also applied to
k-d trees (Algorithm~\ref{alg:fit}, line 7).%
\cite{Brown2015kdtree}
But rather than choosing the splitting value at the median
of the observations in the dimension---done in order to
construct balanced k-d trees---the splitting value
is determined randomly.\cite{Brown2015kdtree}

In order to chose a proper splitting value $\beta_X$ is
passed as another parameter to FIT. $\beta_X$ represents
the boundaries for every dimension of the feature space
based on $X$. For each dimension $\beta_X$ contains a
tuple with the minimum and maximum value in the dimension
of all observations in $X$.

$\beta_X[\text{dimension}]$ is passed to a pseudo-random
number generator\footnote{The implementation used in
  Section~\ref{sec:application} utilizes Python 3.6's
  random library.\cite[chapter 9.6]{python}}
generating a random value so that
$lower(\beta_X[\text{dimension}]) \leq \text{random number}
\leq upper(\beta_X[\text{dimension}])$
(Algorithm~\ref{alg:fit}, line 8).

Afterwards $X$, $y$, $\beta_X$ are split into two new
disjoint partitions and FIT is recursively applied to both
(Algorithm~\ref{alg:fit}, lines 10ff).

Since $\tau_h$ is defined, the maximum amount of nodes a
Tree can have is $2^{\tau_h + 1} - 1$, if the Tree is
perfectly balanced.\cite[chapter 16.1]{Teschl} For each
node FIT is called, so building a Tree has a worst case
time complexity of:
\begin{align}
  \mathcal{O}((2^{\tau_h + 1} -1)*\mathcal{O}(\text{FIT})).
\end{align}
$\mathcal{O}(\text{FIT})$ is determined by the size of $X$%
---since $X$ has to be iterated in order to split it---and
by $\mathcal{O}(\gamma)$. That said, a single FIT
operation would have a worst case time complexity of:
\begin{align}
  \mathcal{O}(|X| + \mathcal{O}(\gamma)),
\end{align} which means the time complexity of the whole
fitting process is:
\begin{align}
  \mathcal{O}((2^{\tau_h + 1} - 1) *
  (|X| + \mathcal{O}(\gamma)))
\label{eq:O_fit}
\end{align}

% algorithm {{{
\begin{algorithm}
  \caption{: FIT($\Theta, X, y, h, \beta_X, \gamma,
    \tau_{l}, \tau_{|X|}, \tau_{h}$)}%
  \label{alg:fit}
  A Tree's FIT operation.

  Inputs:

    \begin{tabu}{llX}
    $\Theta$ &$-$ &a pointer to a Nil node; initially
      pointing to the root node of an empty Tree,\\
    $X$ &$-$ &input data,\\
    $y$ &$-$ &labels of X,\\
    $h$ &$-$ &height of the Tree; initially $h = 0$,\\
    $\beta_X$ &$-$ &lower and upper boundaries of every
      dimension of X,\\
    $\gamma$ &$-$ &function returning a classifier and its
      loss,\\
    $\tau_{l}$ &$-$ &loss threshold,\\
    $\tau_{|X|}$ &$-$ &threshold for the size of X,\\
    $\tau_{h}$ &$-$ &height limit of the Tree
    \end{tabu}

  Output: void

  \noindent\rule{\linewidth}{0.4pt}

  \begin{algorithmic}[1]
    \STATE classifier, loss $\leftarrow \gamma(X, y)$
    \IF{loss $\leq \tau_{l}$}
      \STATE $\Theta \leftarrow$ LEAF(\TRUE, classifier,
         $X$, $y$)
    \ELSIF{$h > \tau_{h}$ \OR $|X| < \tau_{|X|}$ \OR
        loss $> \tau_{l}$}
      \STATE $\Theta \leftarrow$ LEAF(\FALSE, classifier,
        $X$, $y$)
    \ELSE
      \STATE dimension $\leftarrow h$ mod $|X[0]|$
      \STATE split $\leftarrow$ RANDOM($\beta_X[
        \text{dimension}]$)
      \STATE $\Theta \leftarrow$ NODE(split, NIL, NIL)
      \STATE split $X$, $y$ and $\beta_X$ into
        $X'$, $X''$, $y'$, $y''$, $\beta_X'$, $\beta_X''$
      \STATE FIT($\Theta$.left, $X'$, $y'$, $h + 1$,
        $\beta_X'$, \dots)
      \STATE FIT($\Theta$.right, $X''$, $y''$, $h + 1$,
        $\beta_X''$, \dots)
    \ENDIF
  \end{algorithmic}
\end{algorithm}
% }}}

\begin{figure*}
  \begin{subfigure}[b]{\textwidth}
    \centering
    \begin{tikzpicture}
  \datavisualization[
    scientific axes = clean,
    x axis={label={$x_0$}},
    y axis={label={$x_1$}},
    visualize as smooth line/.list={
      split0,split1,split2,split3
    },
    visualize as scatter/.list={0,1},
    0={label in legend={text=label 0},
      style={mark=o, visualizer color=red}},
    1={label in legend={text=label 1},
      style={mark=o, visualizer color=blue}},
    split0={label in legend={text=split $x_0$},
      style={dashed}},
    split1={label in legend={text=split $x_1$}},
    split3={style={dashed}},
  ]
  data {
    x,  y,  set
    0,  0,  0
    0.1,0,  0
    0.2,0.1,0
    0.2,0.3,0
    0.1,0.2,0
    1,  0,  0
    0.4,0.4,1
    1,  1,  1
    0.5,0,  split0
    0.5,1,  split0
    0,  0.5,split1
    0.5,0.5,split1
    0.5,0.2,split2
    1,  0.2,split2
    0.3,0,  split3
    0.3,0.5,split3
  };
\end{tikzpicture}

    \caption{Scatterplot showing the observations and the
      splits done by the FIT operation.}
    \label{fig:scatter_example}
  \end{subfigure}
  \begin{subfigure}[b]{\textwidth}
    \centering
    \scalebox{0.9}{\def\Node#1#2{
  \node[
    rectangle split,
    rectangle split parts=3,
    rectangle split horizontal,
    draw
  ] (#1) {#2\nodepart{two}\nodepart{three}};
}
\def\NodeR#1#2#3{
  \node[
    rectangle split,
    rectangle split parts=3,
    rectangle split horizontal,
    draw,
    #3
  ] (#1) {#2};
}
\def\LeafR#1#2#3{
  \node[
    rectangle split,
    rectangle split parts=3,
    rectangle split horizontal,
    rounded corners,
    draw,
    #3
   ] (#1) {#2};
}
\def\l#1#2{
  \draw[->] ($(#1.two split)!.5!(#1.text split)$)
    |- ($(#1.south)!.5!(#2.north)$) -| (#2.north);
}
\def\r#1#2{
  \draw[->] ($(#1.two split)!.5!(#1.east)$)
    |- ($(#1.south)!.5!(#2.north)$) -| (#2.north);
}
\def\D#1#2#3{
  \matrix[
    ampersand replacement=\&,
    matrix of nodes,
    left delimiter={[},
    right delimiter={]},
    outer ysep=3pt,
    #2
  ] (#1) {#3};
}

\begin{tikzpicture}[
  every left delimiter/.style={xshift=2ex},
  every right delimiter/.style={xshift=-2ex},
]
  % for margin
  \node at(0,1) {};

  \Node{root}{0.5}
    \NodeR{al}{0.5}{below left=0.5 and 2 of root}
      \NodeR{bl}{0.3}{below left=0.5 of al}
        \LeafR{lb}{true}{below left=0.5 and 0.1 of bl}
          \D{lbx}{below left=0.5 and -.75 of lb} {
            0   \& 0   \\
            0.1 \& 0   \\
            0.2 \& 0.1 \\
            0.2 \& 0.3 \\
            0.1 \& 0.2 \\
          }
          \D{lby}{below right=0.5 and .15 of lb} {
            0 \\
            0 \\
            0 \\
            0 \\
            0 \\
          }
        \LeafR{lc}{false}{below right=.5 and 0.1 of bl}
          \D{lcx}{below left=0.5 and -.75 of lc} {
            0.4 \& 0.4 \\
          }
          \D{lcy}{below right=.5 and .15 of lc}{
            1 \\
          }
      \LeafR{la}{false}{below right=0.5 of al}
        \D{lax}{below left=.3 and -.75 of la}{
          \textcolor{white}{1} \&
          \textcolor{white}{1}\\
        }
        \D{lay}{below right=.3 and .15 of la}{
          \textcolor{white}{1}\\
        }
    \NodeR{ar}{0.2}{below right=0.5 and 2 of root}
      \LeafR{ld}{false}{below left=0.5 of ar}
        \D{ldx}{below left=.3 and -.75 of ld} {
          1 \& 0 \\
        }
        \D{ldy}{below right=.3 and .15 of ld} {0 \\}
      \LeafR{le}{false}{below right=0.5 of ar}
        \D{lex}{below left=.3 and -.75 of le} {
          1 \& 1 \\
        }
        \D{ley}{below right=.3 and .15 of le} {1 \\}

  \l{root}{al}
    \l{al}{bl}
      \l{bl}{lb}
        \l{lb}{lbx}
        \r{lb}{lby}
      \r{bl}{lc}
        \l{lc}{lcx}
        \r{lc}{lcy}
    \r{al}{la}
      \l{la}{lax}
      \r{la}{lay}
  \r{root}{ar}
    \l{ar}{ld}
      \l{ld}{ldx}
      \r{ld}{ldy}
    \r{ar}{le}
      \l{le}{lex}
      \r{le}{ley}

  \node[
    rectangle split,
    rectangle split parts=3,
    rectangle split horizontal,
    draw,
    below=3.5 of ar
  ] (nd) {};
  \node[right] at (nd.east) {Node};

  \node[
    rectangle split,
    rectangle split parts=3,
    rectangle split horizontal,
    rounded corners,
    draw,
    below=.5 of nd
   ] (lf) {};
   \node[right] at (lf.east) {Leaf};
\end{tikzpicture}
}
    \caption{The structure of the Tree generated by FIT.}
    \label{fig:tree_example}
  \end{subfigure}
  \caption{Example of FIT applied to the dataset seen in
    Figure~\ref{fig:scatter_example}. $\gamma$ simply
    computes the probability of each label in $y$ and
    returns a function returning the label with the
    maximum probability and as loss one minus the maximum
    probability. The thresholds are: $\tau_l = 1$,
    $\tau_{|X|} = 2$. $\tau_h$ can be any integer above
    2.
  }
  \label{fig:fit_example}
\end{figure*}


\subsection{The PREDICT operation}

The PREDICT operation traverses a Tree until it encounters
a Leaf. If the Leaf is active a label to a provided
observation $x$ is returned by the predictor
property of the Leaf, otherwise $\Lambda$ is returned.

$\Lambda$ must not be an element of the label set.

\begin{algorithm}
  \caption{: PREDICT($\Theta, x, h$)}%
  \label{alg:pred}
  Inputs:

    \begin{tabu}{llX}
    $\Theta$ &$-$ &a Tree node; initially pointing to the
      root of the Tree,\\
    $x$ &$-$ &an observation,\\
    $h$ &$-$ &height of the Tree; initially $h = 0$
    \end{tabu}

  Output: the predicted label or $\Lambda$

  \noindent\rule{\linewidth}{0.4pt}

  \begin{algorithmic}[1]
    \IF{TYPE($\Theta$) is Node}
      \STATE dimension $\leftarrow h$ mod $|x|$
      \IF{$x[dimension] \leq \Theta$.split}
        \STATE PREDICT($\Theta$.left, $x$, $h + 1$)
      \ELSE
        \STATE PREDICT($\Theta$.right, $x$, $h + 1$)
      \ENDIF
    \ELSIF{$\Theta$.active}
      \RETURN $\Theta$.predictor($x$)
    \ELSE
      \RETURN $\Lambda$
    \ENDIF
  \end{algorithmic}
\end{algorithm}

