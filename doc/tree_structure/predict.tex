\subsection{The PREDICT operation}

The PREDICT operation traverses a Tree until it encounters
a Leaf. If the Leaf is active a label to a provided
observation $x$ is returned by the classifier
property of the Leaf, otherwise $\Lambda$ is returned
(Algorithm~\ref{alg:pred}).

$\Lambda$ must not be an element of the label set.

Since the PCF does not predict on its own but instead uses
other classifier instances for the actual prediction the
PCF is a Meta Classifier.\cite[chapter 4.6]{pymvpa}

PREDICT is fairly similar to the search operation of a
binary tree, except for the type distinction and the
prediction.\cite[chapter 12.2]{Cormen} Therefore, PREDICT
has a worst case time complexity of $\mathcal{O}(\tau_h +
\mathcal{O}(\text{classifier}))$.

\begin{algorithm}
  \caption{: PREDICT($\Theta, x, h$)}%
  \label{alg:pred}
  A Tree's PREDICT operation.

  Inputs:

    \begin{tabu}{llX}
    $\Theta$ &$-$ &a Tree node; initially pointing to the
      root of the Tree,\\
    $x$ &$-$ &an observation,\\
    $h$ &$-$ &height of the Tree; initially $h = 0$
    \end{tabu}

  Output: the predicted label or $\Lambda$

  \noindent\rule{\linewidth}{0.4pt}

  \begin{algorithmic}[1]
    \IF{TYPE($\Theta$) is Node}
      \STATE dimension $\leftarrow h$ mod $|x|$
      \IF{$x[\text{dimension}] \leq \Theta$.split}
        \STATE PREDICT($\Theta$.left, $x$, $h + 1$)
      \ELSE
        \STATE PREDICT($\Theta$.right, $x$, $h + 1$)
      \ENDIF
    \ELSIF{$\Theta$.active}
      \RETURN $\Theta$.classifier($x$)
    \ELSE
      \RETURN $\Lambda$
    \ENDIF
  \end{algorithmic}
\end{algorithm}
