\begin{abstract}
  This paper is concerned with a paradigm I call Partial
  Classification. It is based on the premise that for a
  given classification problem a solution must fulfill a
  certain quality criteria not always achievable by
  non-partial classification approaches. For some of these
  classification problems a partial solution achieving the
  quality criteria is more useful than a non-partial
  solution not fulfilling the criteria. A partial solution
  is a classifier that either predicts an observation or
  returns nothing if the corresponding label is not certain
  enough.

  The whole idea behind Partial Classification opposes
  the classical, non-partial classification approach:
  rather than increasing the quality of a classifier,
  the quality --- which must be achieved --- is set
  beforehand and the space in which the classifier can
  predict with the given quality needs to be increased
  instead.

  This paper proposes a method realizing Partial
  Classification called Partial Classification Forest
  (PCF). The PCF is a Monte Carlo based ensemble method.
  It is a Meta Classifier generating instances of a k-d
  tree like structure in order to partition the feature
  space of a given dataset. The PCF predicts on those
  partitions in which it achieves the given quality
  threshold.
\end{abstract}

\begin{IEEEkeywords}
  Supervised Machine Learning, Partial Classification,
  Monte Carlo Method, Meta Classifier
\end{IEEEkeywords}

