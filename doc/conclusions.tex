\section{Conclusions}
\label{sec:conclusions}

In the Introduction I described what I mean by
Partial Classification. The whole concept is based on the
condition that a classifier needs to maintain a certain
quality in its predictions for a given classification
problem. If it can not do this, it is not regarded a
solution.

Partial Classification is a way to still find classifiers
that partially solve a classification problem not solvable
as a whole. A partial classifier only predicts the label
of an observation if it lays inside a partition it can
classify. Otherwise it returns nothing, indicating that
the classifier is not sure which label the observation has.

This paper provides a description of such a partial
classifier, the Partial Classification Forest. It
describes the core structures and operations of the PCF and
shows its use on an artificial problem designed to display
the use of it over other, non-partial classifiers.

Also stated in the Introduction, this paper has some
shortcomings in research and empirical tests regarding the
PCF, due to a lack of time and no complete, fast
implementation. A list of these shortcomings is also
provided in the Introduction. The next step will be to make
a fast and complete implementation before proceeding with
the research and the empirical tests.
