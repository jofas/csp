\section{Further optimizations and additional features}
\label{sec:oandf}

There are some possible optimizations to the PCF which were
not discussed in the previous Sections. Again it should be
noted that these optimizations are yet untested (see
Introduction).

The proposed optimizations are:

\begin{enumerate}

  \item Weighing partitions.

  \item Densifying a Tree after FIT.

  \item Chosing the dimension a Node has randomly rather
        than cyclic (cmp. Section~\ref{subsec:fit}).

\end{enumerate}

The first optimization would be to weigh partitions. As of
right now, the PCF's PREDICT operation determines $l_{max}$
as the label predicted most (Algorithm~\ref{alg:pcf_pred},
line 5).

This could be further refined with weighing each partition
based on two properties: (\romannumeral 1) the amount
of observations a partition contains and (\romannumeral 2)
its volume. This would result in a weight determined as:
\begin{align}
  \text{weight} = \frac{|\text{partition}.X|}
  {V(\text{partition})}.
\end{align}
A partition with a lot of observations and a small
volume would have a higher weight than one with a small
amount of observations and a high volume, further
increasing the propability that the label predicted by the
partition with the higher weight is determined as
$l_{max}$. So instead of just counting each label predicted
and returning the one predicted most, the weight of each
prediction is summed and the label with the highest sum
will be returned by PREDCIT as $l_{max}$.

The second optimization is densifying a Tree after FIT.
If a Node has two unactive Leaves as children the Node is
unnecessary since either way $\Lambda$ is returned. The
Node could be transformed into an unactive Leaf, reducing
the path length of the branch by one and decreasing the
amount of nodes by two, which decreases the size of the
Tree and therefore the time complexity of PREDICT.

The last possible optimization would be to chose the
splitting  dimension, like the splitting value, at random
rather than cyclic. A Node would have to store the
dimension it is splitting.

In Section~\ref{sec:tests} and in Figure~%
\ref{fig:fit_example} a very simple classifier is returned
by $\gamma$. It only computes the probability for each
label and returns the label with the highest probability
(cmp. Figure~\ref{fig:fit_example}).

A classifier like that does not need to know where the
observations are in the partition which makes it
unnecessary to keep them in a Leaf node as $X$ and $y$
(cmp. Algorithm~\ref{alg:fit}). Instead a dictionary with
every label from the label space where each label is mapped
to the amount of observations inside the partition having
the particular label is enough.

A possible feature would be to provide a second variant of
the PCF which passes this dictionary instead of the
observations to $\gamma$, decreasing the complexity of the
PCF for classifiers which do not need to know where each
observations lays inside the partition.

