\documentclass[12pt]{article}
\usepackage{amsmath}
\usepackage{amssymb}
\usepackage[colorlinks=true, linkcolor=black,citecolor=black]{hyperref} % Links
\usepackage{makeidx} % Indexierung
\usepackage{siunitx}
%\usepackage[ngerman]{babel} % deutsche Sonderzeichen
\usepackage[utf8]{inputenc}
\usepackage{geometry} % Dokumentendesign wie Seiten- oder Zeilenabstand bestimmen
\usepackage[toc,page]{appendix}

% Graphiken
\usepackage{tikz}
\usepackage{pgfplots}
\usepackage{pgfcore}
\usepackage{pgfopts}
\usepackage{pgfornament}
\usepackage{pgf}
\usepackage{ifthen}
\usepackage{booktabs}

% Tabellen
\usepackage{tabu}
\usepackage{longtable}
\usepackage{colortbl} % Tabellen faerben
\usepackage{multirow}
\usepackage{diagbox} % Tabellenzelle diagonal splitten

\usepackage{xcolor} % Farben
\usepackage[framemethod=tikz]{mdframed} % Hintergrunderstellung
\usepackage{enumitem} % Enumerate mit Buchstaben nummerierbar machen
\usepackage{pdfpages}
\usepackage{listings} % Source-Code darstellen
\usepackage{eurosym} % Eurosymbol
\usepackage[square,numbers]{natbib}
\usepackage{here} % figure an richtiger Stelle positionieren
\usepackage{verbatim} % Blockkommentare mit \begin{comment}...\end{comment}
\usepackage{ulem} % \sout{} (durchgestrichener Text)
\usepackage{import}

% BibLaTex
\bibliographystyle{alpha}

% Aendern des Anhangnamens (Seite und Inhaltsverzeichnis)
\renewcommand\appendixtocname{Anhang}
\renewcommand\appendixpagename{Anhang}

% mdframed Style
\mdfdefinestyle{codebox}{
	linewidth=2.5pt,
	linecolor=codebordercolor,
	backgroundcolor=codecolor,
	shadow=true,
	shadowcolor=black!40!white,
	fontcolor=black,
	everyline=true,
}

% Seitenabstaende
\geometry{left=15mm,right=15mm,top=10mm,bottom=20mm}

% TikZ Bibliotheken
\usetikzlibrary{
    arrows,
    arrows.meta,
    decorations,
    backgrounds,
    positioning,
    fit,
    petri,
    shadows,
    datavisualization.formats.functions,
    calc,
    shapes,
    shapes.multipart
}

%\pgfplotsset{width=7cm,compat=1.15}

\definecolor{codecolor}{HTML}{EEEEEE}
\definecolor{codebordercolor}{HTML}{CCCCCC}

% Standardeinstellungen fuer Source-Code
\lstset{
    language=C,
    breaklines=true,
    keepspaces=true,
    keywordstyle=\bfseries\color{green!70!black},
    basicstyle=\ttfamily\color{black},
    commentstyle=\itshape\color{purple},
    identifierstyle=\color{blue},
    stringstyle=\color{orange},
    showstringspaces=false,
    rulecolor=\color{black},
    tabsize=2,
    escapeinside={\%*}{*\%},
}

%\input{libuml}
%\input{liberm}

\title{Logs}

%  Partial classification of a binary labeled dataset
%  (working title)}

%\author{Jonas Fa{\ss}bender}
\date{}

\begin{document}

\maketitle

\begin{itemize}

  \item 31st of October 2018: \\ \\
    Started reading "Isolation Forest" by Liu, Ting and
    Zhou, a paper where they describe the Isolation Forest
    algorithm. The way the algorithm uses an ensemble of
    "iTrees" which randomly partition the feature space
    to isolate outliers is very interesting for me because
    I think I could use the concept not to isolate outliers
    but partitions which are part of $P_s$.\cite{if}
    Instead of the "iTree" structure I want to implement a
    structure based on a kd-tree.

  \item 1st of November 2018: \\ \\
    Started working on a random dataset generator that
    generates a two dimensional dataset containing
    "clean" patches, meaning areas where data points all
    belong to the same class.

  \item 6th of November 2018: \\ \\
    Finished the random dataset generator generating $D$.
    The generator generates a two dimensional normalized
    set of points (normalized meaning:
    $\forall P(x,y) \in D: x, y \in [0,1)$). The points are
    distributed uniformally, so every possible value on the
    plane has the same chance of being selected.

    The generator takes as input:
      \begin{itemize}

        \item The amount of points $D$ should contain
              ($|D|$).

        \item How much percent of the area should be
              "clean". The idea I have is that, while a
              uniformally distributed dataset is impossible
              to classify correctly since it is random, the
              dataset the generator builds contains
              "patches" or areas which are clean, meaning
              every point inside one of these patches has
              the same label. My goal is to find a
              classifier that produces a $P_s$ so the area
              $P_s$ covers comes as close to the percent of
              clean points as possible.

        \item The amount of patches. The clean points are
              further separated in different patches, all
              randomly set into the plane. $P_s$ should be
              coextensive with all patches.

        \item The seed for the random number generator, so
              results can be compared.

      \end{itemize}
    Figure 1 shows a dataset generated by the random
    dataset generator.

  \item 8th of November 2018: \\ \\
    Started implementing the kd-tree.

\end{itemize}

\begin{figure}
  \begin{center}
    \import{./fig/}{20000_20_5_42.pgf}
  \end{center}
  \caption{A dataset generated by the random dataset
        generator. It contains 20000 data points from which
        20 percent are clean. The clean points are
        distributed in 5 patches. The seed was 42.}
\end{figure}

\bibliography{../csp}

\end{document}
