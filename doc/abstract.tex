\begin{abstract}
  This paper is concerned with a paradigm I call Partial
  Classification. It is based on the premise that for a
  given classification problem a solution must fulfill a
  certain quality criteria not always achievable by
  non-partial classification approaches. For some of these
  classification problems a partial solution achieving the
  quality criteria is more useful than a non-partial
  solution not fulfilling the criteria. A partial solution
  is a classifier that either predicts an observation or
  returns nothing if the corresponding label is not certain
  enough.

  This paper proposes such a partial classifier, a Monte
  Carlo based ensemble method called Partial Classification
  Forest (PCF). The PCF is a Meta Classifier generating
  instances of a k-d tree like structure in order to
  partition the feature space of a given dataset. This
  paper describes the structures and operations of the PCF
  before showing its application and comparing it to other
  non-partial classifiers.

  Afterwards further features and optimizations of the PCF
  are discussed. At least conclusions and a roadmap
  concerning the PCF are presented.
\end{abstract}

\begin{IEEEkeywords}
  Supervised Machine Learning, Partial Classification,
  Monte Carlo Method, Meta Classifier
\end{IEEEkeywords}

